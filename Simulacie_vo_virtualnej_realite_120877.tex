%Peter_Sartoris_article_MIP_2022

\documentclass[10pt,slovak,a4paper]{article} 
%document definition

\usepackage[slovak]{babel}
\usepackage[IL2]{fontenc}
\usepackage[utf8]{inputenc}
\usepackage{graphicx}
\usepackage{url}
\usepackage{hyperref}

\usepackage{cite}

\pagestyle{headings}
%%%%%%%%%%%%%%%%%%%%%%%%%%%%%%%%%%%%%%%%%%%%%%%%%%%%%%%%%%%%%%%%%%%%%%%%%%%%%%%%%%%%
\title{Simulácie vo virtuálnej realite s využitím gamifikácie za účelom vzdelávania a prípravných kurzov\thanks{Semestrálny projekt v predmete Metódy inžinierskej práce, ak. rok 2022/2023, vedenie: Ing. Igor Stupavský}}

\author{Peter Sartoris\\[2pt]
	{\small Slovenská technická univerzita v Bratislave}\\
	{\small Fakulta informatiky a informačných technológií}\\
	{\small \texttt{xsartoris@stuba.sk}}
	}

\date{\small 22.október 2022}
%end of title
%%%%%%%%%%%%%%%%%%%%%%%%%%%%%%%%%%%%%%%%%%%%%%%%%%%%%%%%%%%%%%%%%%%%%%%%%%%%%%%%%%%%
\begin{document}

\maketitle

%\begin{abstract}
%\ldots
%\end{abstract}
%Netreba zobrazovať, esteticky to vyzerá lepšie
%%%%%%%%%%%%%%%%%%%%%%%%%%%%%%%%%%%%%%%%%%%%%%%%%%%%%%%%%%%%%%%%%%%%%%%%%%%%%%%%%%%%

\section{Úvod} \label{Abstract}

Virtuálna realita je bez pochýb nástroj novodobej techniky, ktorý dokáže vytvoriť či simulovať rôzne situácie vyzerajúce skutočne, dôveryhodne a autenticky.
V súčasnosti je virtuálna realita pomerne ľahko dostupná a populárna v hernom priemysle. Čoraz viac sa využíva ale aj vo vedeckých prostrediach, na školách a celkovo v živote ľudí. 
Náplňou týchto hier však nie je len zábava, je to hlavne šikovná interaktívna metóda vzdelávania.
Tento článok hovorí o tom, kde sa tieto metódy gamifikácie uplatňujú [~\ref{}], kedy a prečo by sa mali (nemali) uplatňovať [~\ref{Reason}] a ako by sa mali uplatňovať [~\ref{}]. 
Ak je totiž gamifikácia v simulačnom prostredí  virtuálnej reality využitá správne, motivuje používateľov zlepšovať svoje zručnosti v tom , čo sa práve učia.

%%%%%%%%%%%%%%%%%%%%%%%%%%%%%%%%%%%%%%%%%%%%%%%%%%%%%%%%%%%%%%%%%%%%%%%%%%%%%%%%%%%%

\section{Pojem gamifikácia} \label{Gamification}

Gamifikáciou sa označuje moderný termín uplatňovania (implementácie) herných prvkov a princípov (mechaník) do neherného sveta. Gamifikácia sa stala v dnešnom svete veľmi rozšírená a 
populárna najmä kvôli vývoju techniky a rozvoju používania interaktívnych metód. Účely gamifikácie sú rôzne - od podpory do riešenia istého problému, motivácie do vzdelávania až po
 marketingové kampane. Cieľovou skupinou môže byť takmer ktokoľvek, avšak väčšinou sú ňou študenti. 


%%%%%%%%%%%%%%%%%%%%%%%%%%%%%%%%%%%%%%%%%%%%%%%%%%%%%%%%%%%%%%%%%%%%%%%%%%%%%%%%%%%%

\section{Virtuálna realita a simulácie} \label{Simulations}

INPUT TEXT 
%%%%%%%%%%%%%%%%%%%%%%%%%%%%%%%%%%%%%%%%%%%%%%%%%%%%%%%%%%%%%%%%%%%%%%%%%%%%%%%%%%%%

\section{Použitia gamifikácie v simuláciách} \label{Uses}

INPUT TEXT
%%%%%%%%%%%%%%%%%%%%%%%%%%%%%%%%%%%%%%%%%%%%%%%%%%%%%%%%%%%%%%%%%%%%%%%%%%%%%%%%%%%%

\section{Zoznam oblastí, kde sa simulácie využívajú na edukačné účely} \label{List}

INPUT TEXT
%%%%%%%%%%%%%%%%%%%%%%%%%%%%%%%%%%%%%%%%%%%%%%%%%%%%%%%%%%%%%%%%%%%%%%%%%%%%%%%%%%%%

\section{Kedy a prečo použiť gamifikáciu?} \label{Reason}

INPUT TEXT
%%%%%%%%%%%%%%%%%%%%%%%%%%%%%%%%%%%%%%%%%%%%%%%%%%%%%%%%%%%%%%%%%%%%%%%%%%%%%%%%%%%%

\section{Motivácia štatistikami a levelmi} \label{Motivation}

INPUT TEXT
%%%%%%%%%%%%%%%%%%%%%%%%%%%%%%%%%%%%%%%%%%%%%%%%%%%%%%%%%%%%%%%%%%%%%%%%%%%%%%%%%%%%

\section{Zhodnotenie} \label{Evaluation}

INPUT TEXT
%%%%%%%%%%%%%%%%%%%%%%%%%%%%%%%%%%%%%%%%%%%%%%%%%%%%%%%%%%%%%%%%%%%%%%%%%%%%%%%%%%%%
%%%%%%%%%%%%%%%%%%%%%%%%%%%%%%%%%%%%%%%%%%%%%%%%%%%%%%%%%%%%%%%%%%%%%%%%%%%%%%%%%%%%

% týmto sa generuje zoznam literatúry z obsahu súboru literatura.bib podľa toho, na čo sa v článku odkazujete
%\bibliography{literature_article}
%\bibliographystyle{plain}
%%%%%%%%%%%%%%%%%%%%%%%%%%%%%%%%%%%%%%%%%%%%%%%%%%%%%%%%%%%%%%%%%%%%%%%%%%%%%%%%%%%%
\end{document}
%%%%%%%%%%%%%%%%%%%%%%%%%%%%%%%%%%%%%%%%%%%%%%%%%%%%%%%%%%%%%%%%%%%%%%%%%%%%%%%%%%%%
